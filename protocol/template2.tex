\documentclass[12pt,a4paper,titlepage,oneside]{article}

\usepackage{dideProtocol}
\sloppy


\exercise{Lab Exercise II}

% enter your data here
\authors{
  Andreas Ciachi, Matr. Nr. 1029176 \par
  {\small e1029176@student.tuwien.ac.at} \par
}

\begin{document}

\maketitle

\section*{Simulation}

\begin{figure}[ht!]
  \centering
  \framebox[\linewidth]{
    \rotatebox{30}{Insert your screenshot here.} 
  }
  % The screenshot should span from the first command on the internal interface
  % to the last command on the external interface

  % \includegraphics[width=1.0\linewidth]{your filename here}
  \caption{Screenshot of the simulation showing the operation of the LCD
    controller}
\end{figure}

\section*{Measurement}

\begin{figure}[ht!]
  \centering
  \framebox[\linewidth]{
    \rotatebox{30}{Insert your screenshot here.} 
  }
  % The screenshot should span from the second instruction received on the internal
  % interface to the command generated on the external interface
  
  % \includegraphics[width=1.0\linewidth]{your filename here}
  \caption{Screenshot of the SignalTap logic analyzer measurement}
\end{figure}

\begin{qa}
  \question{How did you configure the trigger for the above measurement? If it
    is clear from a screenshot, you may answer with a figure.} \answer{...}

\end{qa}

\begin{qa}
  \question{Does the SignlTap LA provide corresponding acquisition modes for
    each acquisition mode of the Agilent LA in the lab?
    What are the differences?}
  \answer{...}
\end{qa}

\begin{qa}
  \question{What resources on the FPGA and board do you need to perform a
    measurement with the Agilent LA and what for the SignalTap LA?}
  \answer{...}
\end{qa}

\begin{qa}
  \question{(For SignalTap only.) Assuming you need a very low sampling frequency
    $f_s$ to capture a long time window.
    How would you define a trigger that reliably stops acquisition when a glitch
    of length $f_s^{-1}/10$ occurs on a signal (Hint: You can assume the glitch
    is long enough to serve as a valid clock pulse for a flip flop.
    Also, you may ignore synchronization issues.)?}
  \answer{...}
\end{qa}


\clearpage
\section*{Feedback \& Comments}
\input{feedback}
 
\end{document}
